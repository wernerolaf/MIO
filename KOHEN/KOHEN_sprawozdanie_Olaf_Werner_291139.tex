
    




    
\documentclass[11pt]{article}

    
    \usepackage[breakable]{tcolorbox}
    \tcbset{nobeforeafter} % prevents tcolorboxes being placing in paragraphs
    \usepackage{float}
    \floatplacement{figure}{H} % forces figures to be placed at the correct location
    
    \usepackage[T1]{fontenc}
    % Nicer default font (+ math font) than Computer Modern for most use cases
    \usepackage{mathpazo}

    % Basic figure setup, for now with no caption control since it's done
    % automatically by Pandoc (which extracts ![](path) syntax from Markdown).
    \usepackage{graphicx}
    % We will generate all images so they have a width \maxwidth. This means
    % that they will get their normal width if they fit onto the page, but
    % are scaled down if they would overflow the margins.
    \makeatletter
    \def\maxwidth{\ifdim\Gin@nat@width>\linewidth\linewidth
    \else\Gin@nat@width\fi}
    \makeatother
    \let\Oldincludegraphics\includegraphics
    % Set max figure width to be 80% of text width, for now hardcoded.
    \renewcommand{\includegraphics}[1]{\Oldincludegraphics[width=.8\maxwidth]{#1}}
    % Ensure that by default, figures have no caption (until we provide a
    % proper Figure object with a Caption API and a way to capture that
    % in the conversion process - todo).
    \usepackage{caption}
    \DeclareCaptionLabelFormat{nolabel}{}
    \captionsetup{labelformat=nolabel}

    \usepackage{adjustbox} % Used to constrain images to a maximum size 
    \usepackage{xcolor} % Allow colors to be defined
    \usepackage{enumerate} % Needed for markdown enumerations to work
    \usepackage{geometry} % Used to adjust the document margins
    \usepackage{amsmath} % Equations
    \usepackage{amssymb} % Equations
    \usepackage{textcomp} % defines textquotesingle
    % Hack from http://tex.stackexchange.com/a/47451/13684:
    \AtBeginDocument{%
        \def\PYZsq{\textquotesingle}% Upright quotes in Pygmentized code
    }
    \usepackage{upquote} % Upright quotes for verbatim code
    \usepackage{eurosym} % defines \euro
    \usepackage[mathletters]{ucs} % Extended unicode (utf-8) support
    \usepackage[utf8x]{inputenc} % Allow utf-8 characters in the tex document
    \usepackage{fancyvrb} % verbatim replacement that allows latex
    \usepackage{grffile} % extends the file name processing of package graphics 
                         % to support a larger range 
    % The hyperref package gives us a pdf with properly built
    % internal navigation ('pdf bookmarks' for the table of contents,
    % internal cross-reference links, web links for URLs, etc.)
    \usepackage{hyperref}
    \usepackage{longtable} % longtable support required by pandoc >1.10
    \usepackage{booktabs}  % table support for pandoc > 1.12.2
    \usepackage[inline]{enumitem} % IRkernel/repr support (it uses the enumerate* environment)
    \usepackage[normalem]{ulem} % ulem is needed to support strikethroughs (\sout)
                                % normalem makes italics be italics, not underlines
    \usepackage{mathrsfs}
    

    
    % Colors for the hyperref package
    \definecolor{urlcolor}{rgb}{0,.145,.698}
    \definecolor{linkcolor}{rgb}{.71,0.21,0.01}
    \definecolor{citecolor}{rgb}{.12,.54,.11}

    % ANSI colors
    \definecolor{ansi-black}{HTML}{3E424D}
    \definecolor{ansi-black-intense}{HTML}{282C36}
    \definecolor{ansi-red}{HTML}{E75C58}
    \definecolor{ansi-red-intense}{HTML}{B22B31}
    \definecolor{ansi-green}{HTML}{00A250}
    \definecolor{ansi-green-intense}{HTML}{007427}
    \definecolor{ansi-yellow}{HTML}{DDB62B}
    \definecolor{ansi-yellow-intense}{HTML}{B27D12}
    \definecolor{ansi-blue}{HTML}{208FFB}
    \definecolor{ansi-blue-intense}{HTML}{0065CA}
    \definecolor{ansi-magenta}{HTML}{D160C4}
    \definecolor{ansi-magenta-intense}{HTML}{A03196}
    \definecolor{ansi-cyan}{HTML}{60C6C8}
    \definecolor{ansi-cyan-intense}{HTML}{258F8F}
    \definecolor{ansi-white}{HTML}{C5C1B4}
    \definecolor{ansi-white-intense}{HTML}{A1A6B2}
    \definecolor{ansi-default-inverse-fg}{HTML}{FFFFFF}
    \definecolor{ansi-default-inverse-bg}{HTML}{000000}

    % commands and environments needed by pandoc snippets
    % extracted from the output of `pandoc -s`
    \providecommand{\tightlist}{%
      \setlength{\itemsep}{0pt}\setlength{\parskip}{0pt}}
    \DefineVerbatimEnvironment{Highlighting}{Verbatim}{commandchars=\\\{\}}
    % Add ',fontsize=\small' for more characters per line
    \newenvironment{Shaded}{}{}
    \newcommand{\KeywordTok}[1]{\textcolor[rgb]{0.00,0.44,0.13}{\textbf{{#1}}}}
    \newcommand{\DataTypeTok}[1]{\textcolor[rgb]{0.56,0.13,0.00}{{#1}}}
    \newcommand{\DecValTok}[1]{\textcolor[rgb]{0.25,0.63,0.44}{{#1}}}
    \newcommand{\BaseNTok}[1]{\textcolor[rgb]{0.25,0.63,0.44}{{#1}}}
    \newcommand{\FloatTok}[1]{\textcolor[rgb]{0.25,0.63,0.44}{{#1}}}
    \newcommand{\CharTok}[1]{\textcolor[rgb]{0.25,0.44,0.63}{{#1}}}
    \newcommand{\StringTok}[1]{\textcolor[rgb]{0.25,0.44,0.63}{{#1}}}
    \newcommand{\CommentTok}[1]{\textcolor[rgb]{0.38,0.63,0.69}{\textit{{#1}}}}
    \newcommand{\OtherTok}[1]{\textcolor[rgb]{0.00,0.44,0.13}{{#1}}}
    \newcommand{\AlertTok}[1]{\textcolor[rgb]{1.00,0.00,0.00}{\textbf{{#1}}}}
    \newcommand{\FunctionTok}[1]{\textcolor[rgb]{0.02,0.16,0.49}{{#1}}}
    \newcommand{\RegionMarkerTok}[1]{{#1}}
    \newcommand{\ErrorTok}[1]{\textcolor[rgb]{1.00,0.00,0.00}{\textbf{{#1}}}}
    \newcommand{\NormalTok}[1]{{#1}}
    
    % Additional commands for more recent versions of Pandoc
    \newcommand{\ConstantTok}[1]{\textcolor[rgb]{0.53,0.00,0.00}{{#1}}}
    \newcommand{\SpecialCharTok}[1]{\textcolor[rgb]{0.25,0.44,0.63}{{#1}}}
    \newcommand{\VerbatimStringTok}[1]{\textcolor[rgb]{0.25,0.44,0.63}{{#1}}}
    \newcommand{\SpecialStringTok}[1]{\textcolor[rgb]{0.73,0.40,0.53}{{#1}}}
    \newcommand{\ImportTok}[1]{{#1}}
    \newcommand{\DocumentationTok}[1]{\textcolor[rgb]{0.73,0.13,0.13}{\textit{{#1}}}}
    \newcommand{\AnnotationTok}[1]{\textcolor[rgb]{0.38,0.63,0.69}{\textbf{\textit{{#1}}}}}
    \newcommand{\CommentVarTok}[1]{\textcolor[rgb]{0.38,0.63,0.69}{\textbf{\textit{{#1}}}}}
    \newcommand{\VariableTok}[1]{\textcolor[rgb]{0.10,0.09,0.49}{{#1}}}
    \newcommand{\ControlFlowTok}[1]{\textcolor[rgb]{0.00,0.44,0.13}{\textbf{{#1}}}}
    \newcommand{\OperatorTok}[1]{\textcolor[rgb]{0.40,0.40,0.40}{{#1}}}
    \newcommand{\BuiltInTok}[1]{{#1}}
    \newcommand{\ExtensionTok}[1]{{#1}}
    \newcommand{\PreprocessorTok}[1]{\textcolor[rgb]{0.74,0.48,0.00}{{#1}}}
    \newcommand{\AttributeTok}[1]{\textcolor[rgb]{0.49,0.56,0.16}{{#1}}}
    \newcommand{\InformationTok}[1]{\textcolor[rgb]{0.38,0.63,0.69}{\textbf{\textit{{#1}}}}}
    \newcommand{\WarningTok}[1]{\textcolor[rgb]{0.38,0.63,0.69}{\textbf{\textit{{#1}}}}}
    
    
    % Define a nice break command that doesn't care if a line doesn't already
    % exist.
    \def\br{\hspace*{\fill} \\* }
    % Math Jax compatibility definitions
    \def\gt{>}
    \def\lt{<}
    \let\Oldtex\TeX
    \let\Oldlatex\LaTeX
    \renewcommand{\TeX}{\textrm{\Oldtex}}
    \renewcommand{\LaTeX}{\textrm{\Oldlatex}}
    % Document parameters
    % Document title
    \title{KOHEN\_sprawozdanie\_Olaf\_Werner\_291139}
    
    
    
    
    
% Pygments definitions
\makeatletter
\def\PY@reset{\let\PY@it=\relax \let\PY@bf=\relax%
    \let\PY@ul=\relax \let\PY@tc=\relax%
    \let\PY@bc=\relax \let\PY@ff=\relax}
\def\PY@tok#1{\csname PY@tok@#1\endcsname}
\def\PY@toks#1+{\ifx\relax#1\empty\else%
    \PY@tok{#1}\expandafter\PY@toks\fi}
\def\PY@do#1{\PY@bc{\PY@tc{\PY@ul{%
    \PY@it{\PY@bf{\PY@ff{#1}}}}}}}
\def\PY#1#2{\PY@reset\PY@toks#1+\relax+\PY@do{#2}}

\expandafter\def\csname PY@tok@w\endcsname{\def\PY@tc##1{\textcolor[rgb]{0.73,0.73,0.73}{##1}}}
\expandafter\def\csname PY@tok@c\endcsname{\let\PY@it=\textit\def\PY@tc##1{\textcolor[rgb]{0.25,0.50,0.50}{##1}}}
\expandafter\def\csname PY@tok@cp\endcsname{\def\PY@tc##1{\textcolor[rgb]{0.74,0.48,0.00}{##1}}}
\expandafter\def\csname PY@tok@k\endcsname{\let\PY@bf=\textbf\def\PY@tc##1{\textcolor[rgb]{0.00,0.50,0.00}{##1}}}
\expandafter\def\csname PY@tok@kp\endcsname{\def\PY@tc##1{\textcolor[rgb]{0.00,0.50,0.00}{##1}}}
\expandafter\def\csname PY@tok@kt\endcsname{\def\PY@tc##1{\textcolor[rgb]{0.69,0.00,0.25}{##1}}}
\expandafter\def\csname PY@tok@o\endcsname{\def\PY@tc##1{\textcolor[rgb]{0.40,0.40,0.40}{##1}}}
\expandafter\def\csname PY@tok@ow\endcsname{\let\PY@bf=\textbf\def\PY@tc##1{\textcolor[rgb]{0.67,0.13,1.00}{##1}}}
\expandafter\def\csname PY@tok@nb\endcsname{\def\PY@tc##1{\textcolor[rgb]{0.00,0.50,0.00}{##1}}}
\expandafter\def\csname PY@tok@nf\endcsname{\def\PY@tc##1{\textcolor[rgb]{0.00,0.00,1.00}{##1}}}
\expandafter\def\csname PY@tok@nc\endcsname{\let\PY@bf=\textbf\def\PY@tc##1{\textcolor[rgb]{0.00,0.00,1.00}{##1}}}
\expandafter\def\csname PY@tok@nn\endcsname{\let\PY@bf=\textbf\def\PY@tc##1{\textcolor[rgb]{0.00,0.00,1.00}{##1}}}
\expandafter\def\csname PY@tok@ne\endcsname{\let\PY@bf=\textbf\def\PY@tc##1{\textcolor[rgb]{0.82,0.25,0.23}{##1}}}
\expandafter\def\csname PY@tok@nv\endcsname{\def\PY@tc##1{\textcolor[rgb]{0.10,0.09,0.49}{##1}}}
\expandafter\def\csname PY@tok@no\endcsname{\def\PY@tc##1{\textcolor[rgb]{0.53,0.00,0.00}{##1}}}
\expandafter\def\csname PY@tok@nl\endcsname{\def\PY@tc##1{\textcolor[rgb]{0.63,0.63,0.00}{##1}}}
\expandafter\def\csname PY@tok@ni\endcsname{\let\PY@bf=\textbf\def\PY@tc##1{\textcolor[rgb]{0.60,0.60,0.60}{##1}}}
\expandafter\def\csname PY@tok@na\endcsname{\def\PY@tc##1{\textcolor[rgb]{0.49,0.56,0.16}{##1}}}
\expandafter\def\csname PY@tok@nt\endcsname{\let\PY@bf=\textbf\def\PY@tc##1{\textcolor[rgb]{0.00,0.50,0.00}{##1}}}
\expandafter\def\csname PY@tok@nd\endcsname{\def\PY@tc##1{\textcolor[rgb]{0.67,0.13,1.00}{##1}}}
\expandafter\def\csname PY@tok@s\endcsname{\def\PY@tc##1{\textcolor[rgb]{0.73,0.13,0.13}{##1}}}
\expandafter\def\csname PY@tok@sd\endcsname{\let\PY@it=\textit\def\PY@tc##1{\textcolor[rgb]{0.73,0.13,0.13}{##1}}}
\expandafter\def\csname PY@tok@si\endcsname{\let\PY@bf=\textbf\def\PY@tc##1{\textcolor[rgb]{0.73,0.40,0.53}{##1}}}
\expandafter\def\csname PY@tok@se\endcsname{\let\PY@bf=\textbf\def\PY@tc##1{\textcolor[rgb]{0.73,0.40,0.13}{##1}}}
\expandafter\def\csname PY@tok@sr\endcsname{\def\PY@tc##1{\textcolor[rgb]{0.73,0.40,0.53}{##1}}}
\expandafter\def\csname PY@tok@ss\endcsname{\def\PY@tc##1{\textcolor[rgb]{0.10,0.09,0.49}{##1}}}
\expandafter\def\csname PY@tok@sx\endcsname{\def\PY@tc##1{\textcolor[rgb]{0.00,0.50,0.00}{##1}}}
\expandafter\def\csname PY@tok@m\endcsname{\def\PY@tc##1{\textcolor[rgb]{0.40,0.40,0.40}{##1}}}
\expandafter\def\csname PY@tok@gh\endcsname{\let\PY@bf=\textbf\def\PY@tc##1{\textcolor[rgb]{0.00,0.00,0.50}{##1}}}
\expandafter\def\csname PY@tok@gu\endcsname{\let\PY@bf=\textbf\def\PY@tc##1{\textcolor[rgb]{0.50,0.00,0.50}{##1}}}
\expandafter\def\csname PY@tok@gd\endcsname{\def\PY@tc##1{\textcolor[rgb]{0.63,0.00,0.00}{##1}}}
\expandafter\def\csname PY@tok@gi\endcsname{\def\PY@tc##1{\textcolor[rgb]{0.00,0.63,0.00}{##1}}}
\expandafter\def\csname PY@tok@gr\endcsname{\def\PY@tc##1{\textcolor[rgb]{1.00,0.00,0.00}{##1}}}
\expandafter\def\csname PY@tok@ge\endcsname{\let\PY@it=\textit}
\expandafter\def\csname PY@tok@gs\endcsname{\let\PY@bf=\textbf}
\expandafter\def\csname PY@tok@gp\endcsname{\let\PY@bf=\textbf\def\PY@tc##1{\textcolor[rgb]{0.00,0.00,0.50}{##1}}}
\expandafter\def\csname PY@tok@go\endcsname{\def\PY@tc##1{\textcolor[rgb]{0.53,0.53,0.53}{##1}}}
\expandafter\def\csname PY@tok@gt\endcsname{\def\PY@tc##1{\textcolor[rgb]{0.00,0.27,0.87}{##1}}}
\expandafter\def\csname PY@tok@err\endcsname{\def\PY@bc##1{\setlength{\fboxsep}{0pt}\fcolorbox[rgb]{1.00,0.00,0.00}{1,1,1}{\strut ##1}}}
\expandafter\def\csname PY@tok@kc\endcsname{\let\PY@bf=\textbf\def\PY@tc##1{\textcolor[rgb]{0.00,0.50,0.00}{##1}}}
\expandafter\def\csname PY@tok@kd\endcsname{\let\PY@bf=\textbf\def\PY@tc##1{\textcolor[rgb]{0.00,0.50,0.00}{##1}}}
\expandafter\def\csname PY@tok@kn\endcsname{\let\PY@bf=\textbf\def\PY@tc##1{\textcolor[rgb]{0.00,0.50,0.00}{##1}}}
\expandafter\def\csname PY@tok@kr\endcsname{\let\PY@bf=\textbf\def\PY@tc##1{\textcolor[rgb]{0.00,0.50,0.00}{##1}}}
\expandafter\def\csname PY@tok@bp\endcsname{\def\PY@tc##1{\textcolor[rgb]{0.00,0.50,0.00}{##1}}}
\expandafter\def\csname PY@tok@fm\endcsname{\def\PY@tc##1{\textcolor[rgb]{0.00,0.00,1.00}{##1}}}
\expandafter\def\csname PY@tok@vc\endcsname{\def\PY@tc##1{\textcolor[rgb]{0.10,0.09,0.49}{##1}}}
\expandafter\def\csname PY@tok@vg\endcsname{\def\PY@tc##1{\textcolor[rgb]{0.10,0.09,0.49}{##1}}}
\expandafter\def\csname PY@tok@vi\endcsname{\def\PY@tc##1{\textcolor[rgb]{0.10,0.09,0.49}{##1}}}
\expandafter\def\csname PY@tok@vm\endcsname{\def\PY@tc##1{\textcolor[rgb]{0.10,0.09,0.49}{##1}}}
\expandafter\def\csname PY@tok@sa\endcsname{\def\PY@tc##1{\textcolor[rgb]{0.73,0.13,0.13}{##1}}}
\expandafter\def\csname PY@tok@sb\endcsname{\def\PY@tc##1{\textcolor[rgb]{0.73,0.13,0.13}{##1}}}
\expandafter\def\csname PY@tok@sc\endcsname{\def\PY@tc##1{\textcolor[rgb]{0.73,0.13,0.13}{##1}}}
\expandafter\def\csname PY@tok@dl\endcsname{\def\PY@tc##1{\textcolor[rgb]{0.73,0.13,0.13}{##1}}}
\expandafter\def\csname PY@tok@s2\endcsname{\def\PY@tc##1{\textcolor[rgb]{0.73,0.13,0.13}{##1}}}
\expandafter\def\csname PY@tok@sh\endcsname{\def\PY@tc##1{\textcolor[rgb]{0.73,0.13,0.13}{##1}}}
\expandafter\def\csname PY@tok@s1\endcsname{\def\PY@tc##1{\textcolor[rgb]{0.73,0.13,0.13}{##1}}}
\expandafter\def\csname PY@tok@mb\endcsname{\def\PY@tc##1{\textcolor[rgb]{0.40,0.40,0.40}{##1}}}
\expandafter\def\csname PY@tok@mf\endcsname{\def\PY@tc##1{\textcolor[rgb]{0.40,0.40,0.40}{##1}}}
\expandafter\def\csname PY@tok@mh\endcsname{\def\PY@tc##1{\textcolor[rgb]{0.40,0.40,0.40}{##1}}}
\expandafter\def\csname PY@tok@mi\endcsname{\def\PY@tc##1{\textcolor[rgb]{0.40,0.40,0.40}{##1}}}
\expandafter\def\csname PY@tok@il\endcsname{\def\PY@tc##1{\textcolor[rgb]{0.40,0.40,0.40}{##1}}}
\expandafter\def\csname PY@tok@mo\endcsname{\def\PY@tc##1{\textcolor[rgb]{0.40,0.40,0.40}{##1}}}
\expandafter\def\csname PY@tok@ch\endcsname{\let\PY@it=\textit\def\PY@tc##1{\textcolor[rgb]{0.25,0.50,0.50}{##1}}}
\expandafter\def\csname PY@tok@cm\endcsname{\let\PY@it=\textit\def\PY@tc##1{\textcolor[rgb]{0.25,0.50,0.50}{##1}}}
\expandafter\def\csname PY@tok@cpf\endcsname{\let\PY@it=\textit\def\PY@tc##1{\textcolor[rgb]{0.25,0.50,0.50}{##1}}}
\expandafter\def\csname PY@tok@c1\endcsname{\let\PY@it=\textit\def\PY@tc##1{\textcolor[rgb]{0.25,0.50,0.50}{##1}}}
\expandafter\def\csname PY@tok@cs\endcsname{\let\PY@it=\textit\def\PY@tc##1{\textcolor[rgb]{0.25,0.50,0.50}{##1}}}

\def\PYZbs{\char`\\}
\def\PYZus{\char`\_}
\def\PYZob{\char`\{}
\def\PYZcb{\char`\}}
\def\PYZca{\char`\^}
\def\PYZam{\char`\&}
\def\PYZlt{\char`\<}
\def\PYZgt{\char`\>}
\def\PYZsh{\char`\#}
\def\PYZpc{\char`\%}
\def\PYZdl{\char`\$}
\def\PYZhy{\char`\-}
\def\PYZsq{\char`\'}
\def\PYZdq{\char`\"}
\def\PYZti{\char`\~}
% for compatibility with earlier versions
\def\PYZat{@}
\def\PYZlb{[}
\def\PYZrb{]}
\makeatother


    % For linebreaks inside Verbatim environment from package fancyvrb. 
    \makeatletter
        \newbox\Wrappedcontinuationbox 
        \newbox\Wrappedvisiblespacebox 
        \newcommand*\Wrappedvisiblespace {\textcolor{red}{\textvisiblespace}} 
        \newcommand*\Wrappedcontinuationsymbol {\textcolor{red}{\llap{\tiny$\m@th\hookrightarrow$}}} 
        \newcommand*\Wrappedcontinuationindent {3ex } 
        \newcommand*\Wrappedafterbreak {\kern\Wrappedcontinuationindent\copy\Wrappedcontinuationbox} 
        % Take advantage of the already applied Pygments mark-up to insert 
        % potential linebreaks for TeX processing. 
        %        {, <, #, %, $, ' and ": go to next line. 
        %        _, }, ^, &, >, - and ~: stay at end of broken line. 
        % Use of \textquotesingle for straight quote. 
        \newcommand*\Wrappedbreaksatspecials {% 
            \def\PYGZus{\discretionary{\char`\_}{\Wrappedafterbreak}{\char`\_}}% 
            \def\PYGZob{\discretionary{}{\Wrappedafterbreak\char`\{}{\char`\{}}% 
            \def\PYGZcb{\discretionary{\char`\}}{\Wrappedafterbreak}{\char`\}}}% 
            \def\PYGZca{\discretionary{\char`\^}{\Wrappedafterbreak}{\char`\^}}% 
            \def\PYGZam{\discretionary{\char`\&}{\Wrappedafterbreak}{\char`\&}}% 
            \def\PYGZlt{\discretionary{}{\Wrappedafterbreak\char`\<}{\char`\<}}% 
            \def\PYGZgt{\discretionary{\char`\>}{\Wrappedafterbreak}{\char`\>}}% 
            \def\PYGZsh{\discretionary{}{\Wrappedafterbreak\char`\#}{\char`\#}}% 
            \def\PYGZpc{\discretionary{}{\Wrappedafterbreak\char`\%}{\char`\%}}% 
            \def\PYGZdl{\discretionary{}{\Wrappedafterbreak\char`\$}{\char`\$}}% 
            \def\PYGZhy{\discretionary{\char`\-}{\Wrappedafterbreak}{\char`\-}}% 
            \def\PYGZsq{\discretionary{}{\Wrappedafterbreak\textquotesingle}{\textquotesingle}}% 
            \def\PYGZdq{\discretionary{}{\Wrappedafterbreak\char`\"}{\char`\"}}% 
            \def\PYGZti{\discretionary{\char`\~}{\Wrappedafterbreak}{\char`\~}}% 
        } 
        % Some characters . , ; ? ! / are not pygmentized. 
        % This macro makes them "active" and they will insert potential linebreaks 
        \newcommand*\Wrappedbreaksatpunct {% 
            \lccode`\~`\.\lowercase{\def~}{\discretionary{\hbox{\char`\.}}{\Wrappedafterbreak}{\hbox{\char`\.}}}% 
            \lccode`\~`\,\lowercase{\def~}{\discretionary{\hbox{\char`\,}}{\Wrappedafterbreak}{\hbox{\char`\,}}}% 
            \lccode`\~`\;\lowercase{\def~}{\discretionary{\hbox{\char`\;}}{\Wrappedafterbreak}{\hbox{\char`\;}}}% 
            \lccode`\~`\:\lowercase{\def~}{\discretionary{\hbox{\char`\:}}{\Wrappedafterbreak}{\hbox{\char`\:}}}% 
            \lccode`\~`\?\lowercase{\def~}{\discretionary{\hbox{\char`\?}}{\Wrappedafterbreak}{\hbox{\char`\?}}}% 
            \lccode`\~`\!\lowercase{\def~}{\discretionary{\hbox{\char`\!}}{\Wrappedafterbreak}{\hbox{\char`\!}}}% 
            \lccode`\~`\/\lowercase{\def~}{\discretionary{\hbox{\char`\/}}{\Wrappedafterbreak}{\hbox{\char`\/}}}% 
            \catcode`\.\active
            \catcode`\,\active 
            \catcode`\;\active
            \catcode`\:\active
            \catcode`\?\active
            \catcode`\!\active
            \catcode`\/\active 
            \lccode`\~`\~ 	
        }
    \makeatother

    \let\OriginalVerbatim=\Verbatim
    \makeatletter
    \renewcommand{\Verbatim}[1][1]{%
        %\parskip\z@skip
        \sbox\Wrappedcontinuationbox {\Wrappedcontinuationsymbol}%
        \sbox\Wrappedvisiblespacebox {\FV@SetupFont\Wrappedvisiblespace}%
        \def\FancyVerbFormatLine ##1{\hsize\linewidth
            \vtop{\raggedright\hyphenpenalty\z@\exhyphenpenalty\z@
                \doublehyphendemerits\z@\finalhyphendemerits\z@
                \strut ##1\strut}%
        }%
        % If the linebreak is at a space, the latter will be displayed as visible
        % space at end of first line, and a continuation symbol starts next line.
        % Stretch/shrink are however usually zero for typewriter font.
        \def\FV@Space {%
            \nobreak\hskip\z@ plus\fontdimen3\font minus\fontdimen4\font
            \discretionary{\copy\Wrappedvisiblespacebox}{\Wrappedafterbreak}
            {\kern\fontdimen2\font}%
        }%
        
        % Allow breaks at special characters using \PYG... macros.
        \Wrappedbreaksatspecials
        % Breaks at punctuation characters . , ; ? ! and / need catcode=\active 	
        \OriginalVerbatim[#1,codes*=\Wrappedbreaksatpunct]%
    }
    \makeatother

    % Exact colors from NB
    \definecolor{incolor}{HTML}{303F9F}
    \definecolor{outcolor}{HTML}{D84315}
    \definecolor{cellborder}{HTML}{CFCFCF}
    \definecolor{cellbackground}{HTML}{F7F7F7}
    
    % prompt
    \newcommand{\prompt}[4]{
        \llap{{\color{#2}[#3]: #4}}\vspace{-1.25em}
    }
    

    
    % Prevent overflowing lines due to hard-to-break entities
    \sloppy 
    % Setup hyperref package
    \hypersetup{
      breaklinks=true,  % so long urls are correctly broken across lines
      colorlinks=true,
      urlcolor=urlcolor,
      linkcolor=linkcolor,
      citecolor=citecolor,
      }
    % Slightly bigger margins than the latex defaults
    
    \geometry{verbose,tmargin=1in,bmargin=1in,lmargin=1in,rmargin=1in}
    
    

    \begin{document}
    
    
    \maketitle
    
    

    
    \hypertarget{potwierdzam-samodzielnoux15bux107-powyux17cszej-pracy-oraz-niekorzystanie-przeze-mnie-z-niedozwolonych-ux17aruxf3deux142.-olaf-werner}{%
\section{Potwierdzam samodzielność powyższej pracy oraz niekorzystanie
przeze mnie z niedozwolonych źródeł. Olaf
Werner}\label{potwierdzam-samodzielnoux15bux107-powyux17cszej-pracy-oraz-niekorzystanie-przeze-mnie-z-niedozwolonych-ux17aruxf3deux142.-olaf-werner}}

    \hypertarget{wstux119p}{%
\section{Wstęp}\label{wstux119p}}

    Zbadamy działanie sieci Kohena na dwóch zbiorach danych: MNIST i UCI
HAR. Będziemy testować w zależności od architektury sieci (sześciokątnej
i kwadratowej) oraz funkcji sąsiedztwa: gaussowskiej i minus drugiej
pochodnej funkcji gaussowskiej dalej zwaną ``sombrero''. Sombrero jest
wygaszana 3 razy szybciej. Funkcją wygasającą będzie wykładnicza.
Początkowe wagi neuronów będą losowane jednostajnie na przedziale
{[}-1,1{]} i będą takie same dla wszystkich sieci w ramach danego zbioru
danych. Weżmiemy architekturę 4 na 4 i będziemy trenować na losowym
podzbiorzę o wielkości 1/10 oryginalnego. Będziemy mieli 100 iteracji.

    \hypertarget{mnist}{%
\section{MNIST}\label{mnist}}

    Normalizujemy dane poprzez podzielenie na 255, pomnożenie przez dwa i
odjęcie 1.

    \hypertarget{szeux15bciokux105tna-i-gaussowska}{%
\subsection{Sześciokątna i
gaussowska}\label{szeux15bciokux105tna-i-gaussowska}}

    \hypertarget{klasa-i-bmu-dla-kaux17cdej-obserwacji-testowej}{%
\subsection{Klasa i BMU dla każdej obserwacji
testowej}\label{klasa-i-bmu-dla-kaux17cdej-obserwacji-testowej}}

            
        
    \begin{center}
    \adjustimage{max size={0.9\linewidth}{0.9\paperheight}}{KOHEN_sprawozdanie_Olaf_Werner_291139_files/KOHEN_sprawozdanie_Olaf_Werner_291139_16_1.png}
    \end{center}
    { \hspace*{\fill} \\}
    
    Widzimy że nasza sieć dobrze dopasowywuje się do 0 i 6 reszta klastrów
jest mocno wymieszana.

    \hypertarget{czystoux15bux107-w-zaleux17cnoux15bci-od-neuronu}{%
\subsection{Czystość w zależności od
neuronu}\label{czystoux15bux107-w-zaleux17cnoux15bci-od-neuronu}}

            
        
    \begin{center}
    \adjustimage{max size={0.9\linewidth}{0.9\paperheight}}{KOHEN_sprawozdanie_Olaf_Werner_291139_files/KOHEN_sprawozdanie_Olaf_Werner_291139_19_1.png}
    \end{center}
    { \hspace*{\fill} \\}
    
    Tutaj i w dalszych wykresach czystość rozumiemy jako częstość
występowania najczęściej występującej klasy. Ten wykres potwierdza nasze
przypuszczenia o dobrym dopasowaniu 0 i 6, reszta jest dopasowana lepiej
niż losowo, oprócz neuronu {[}0,3{]} który ma słabe rezultaty.

    \hypertarget{u-macierz}{%
\subsection{U Macierz}\label{u-macierz}}

            
        
    \begin{center}
    \adjustimage{max size={0.9\linewidth}{0.9\paperheight}}{KOHEN_sprawozdanie_Olaf_Werner_291139_files/KOHEN_sprawozdanie_Olaf_Werner_291139_22_1.png}
    \end{center}
    { \hspace*{\fill} \\}
    
    Liczby oznaczają tutaj najczęściej występującą klasę, zaś im ciemniejszy
obszar tym bliżej siebie są dane klasy. Tu widzimy że 0 i 6 są bardzo
dobrze rozdzielane od innych klas. To znaczy że ich neurony są znacząco
inne od innych neuronów. Inne są bardziej do siebie podobne co tłumaczy
ich niższą czystość.

    \hypertarget{szeux15bciokux105tna-i-sombrero}{%
\subsection{Sześciokątna i
sombrero}\label{szeux15bciokux105tna-i-sombrero}}

    \hypertarget{klasa-i-bmu-dla-kaux17cdej-obserwacji-testowej}{%
\subsection{Klasa i BMU dla każdej obserwacji
testowej}\label{klasa-i-bmu-dla-kaux17cdej-obserwacji-testowej}}

           
        
    \begin{center}
    \adjustimage{max size={0.9\linewidth}{0.9\paperheight}}{KOHEN_sprawozdanie_Olaf_Werner_291139_files/KOHEN_sprawozdanie_Olaf_Werner_291139_27_1.png}
    \end{center}
    { \hspace*{\fill} \\}
    
    Mamy mniej klastrów niż klas to jest bardzo źle.

    \hypertarget{czystoux15bux107-w-zaleux17cnoux15bci-od-neuronu}{%
\subsection{Czystość w zależności od
neuronu}\label{czystoux15bux107-w-zaleux17cnoux15bci-od-neuronu}}

            
        
    \begin{center}
    \adjustimage{max size={0.9\linewidth}{0.9\paperheight}}{KOHEN_sprawozdanie_Olaf_Werner_291139_files/KOHEN_sprawozdanie_Olaf_Werner_291139_30_1.png}
    \end{center}
    { \hspace*{\fill} \\}
    
    Oprócz 3 to inne liczby są prawie losowe.

    \hypertarget{u-macierz}{%
\subsection{U Macierz}\label{u-macierz}}

            
        
    \begin{center}
    \adjustimage{max size={0.9\linewidth}{0.9\paperheight}}{KOHEN_sprawozdanie_Olaf_Werner_291139_files/KOHEN_sprawozdanie_Olaf_Werner_291139_33_1.png}
    \end{center}
    { \hspace*{\fill} \\}
    
    Umacierz wyjaśnia co się stało. Neuron liczby 3 jest w miarę odizolowany
od neuronów 4 i 1 które są bardzo blisko siebie co tłumaczy czemu 3 było
w miarę czystę, zaś 4 i 1 nie. Pozostałe neurony są bardzo daleko od
siebie nawzajem i naszych ``działających'' wynika to z tego że sombrero
może dawać ujemne wartości przez co w trakcie uczenia neurony są
``odpychane'' od BMU.

    \hypertarget{kwadratowa-i-gaussowska}{%
\subsection{Kwadratowa i gaussowska}\label{kwadratowa-i-gaussowska}}

    \hypertarget{klasa-i-bmu-dla-kaux17cdej-obserwacji-testowej}{%
\subsection{Klasa i BMU dla każdej obserwacji
testowej}\label{klasa-i-bmu-dla-kaux17cdej-obserwacji-testowej}}

            
        
    \begin{center}
    \adjustimage{max size={0.9\linewidth}{0.9\paperheight}}{KOHEN_sprawozdanie_Olaf_Werner_291139_files/KOHEN_sprawozdanie_Olaf_Werner_291139_38_1.png}
    \end{center}
    { \hspace*{\fill} \\}
    
    Sieć dała podobne rezultaty co ta o architekturzę sześciokątnej czyli 0
i 6 są dobrze separowane, reszta nie jest tak czysta.

    \hypertarget{czystoux15bux107-w-zaleux17cnoux15bci-od-neuronu}{%
\subsection{Czystość w zależności od
neuronu}\label{czystoux15bux107-w-zaleux17cnoux15bci-od-neuronu}}

            
        
    


    
    \begin{center}
    \adjustimage{max size={0.9\linewidth}{0.9\paperheight}}{KOHEN_sprawozdanie_Olaf_Werner_291139_files/KOHEN_sprawozdanie_Olaf_Werner_291139_41_2.png}
    \end{center}
    { \hspace*{\fill} \\}
    
    Tutaj 9 też nigdzie nie jest najczęściej występującą wartością. 6 i 0
też są najlepsze.

    \hypertarget{u-macierz}{%
\subsection{U Macierz}\label{u-macierz}}

    \begin{center}
    \adjustimage{max size={0.9\linewidth}{0.9\paperheight}}{KOHEN_sprawozdanie_Olaf_Werner_291139_files/KOHEN_sprawozdanie_Olaf_Werner_291139_44_0.png}
    \end{center}
    { \hspace*{\fill} \\}
    
    Im ciemniej tym bliżej są siebie neurony. Możemy wyróżnić tu klaster 8
na górze, 0 w lewym centrum. Klastry 6 na dole są od siebie odseparowane
1. Podobnie jak w siatce sześciokątnej im bliżej siebie są różne liczby
tym gorszej czystości są to klastry.

    \hypertarget{kwadratowa-i-sombrero}{%
\subsection{Kwadratowa i sombrero}\label{kwadratowa-i-sombrero}}

    \hypertarget{klasa-i-bmu-dla-kaux17cdej-obserwacji-testowej}{%
\subsection{Klasa i BMU dla każdej obserwacji
testowej}\label{klasa-i-bmu-dla-kaux17cdej-obserwacji-testowej}}

            
        
    \begin{center}
    \adjustimage{max size={0.9\linewidth}{0.9\paperheight}}{KOHEN_sprawozdanie_Olaf_Werner_291139_files/KOHEN_sprawozdanie_Olaf_Werner_291139_49_1.png}
    \end{center}
    { \hspace*{\fill} \\}
    
    Podobnie jak w przypadku sześciokątnej siatki. Mamy tylko 3 klastry.

    \hypertarget{czystoux15bux107-w-zaleux17cnoux15bci-od-neuronu}{%
\subsection{Czystość w zależności od
neuronu}\label{czystoux15bux107-w-zaleux17cnoux15bci-od-neuronu}}

            
        
    
    

    
    \begin{center}
    \adjustimage{max size={0.9\linewidth}{0.9\paperheight}}{KOHEN_sprawozdanie_Olaf_Werner_291139_files/KOHEN_sprawozdanie_Olaf_Werner_291139_52_2.png}
    \end{center}
    { \hspace*{\fill} \\}
    
    3 też zostało najlepiej dopasowane.

    \hypertarget{u-macierz}{%
\subsection{U Macierz}\label{u-macierz}}

    \begin{center}
    \adjustimage{max size={0.9\linewidth}{0.9\paperheight}}{KOHEN_sprawozdanie_Olaf_Werner_291139_files/KOHEN_sprawozdanie_Olaf_Werner_291139_55_0.png}
    \end{center}
    { \hspace*{\fill} \\}
    
    Tutaj neuron {[}2,1{]} został ``odepchnięty'' tak mocno że zmienił
``kolorystykę'' całej u macierzy. Klastry ``działające'' też są od
siebie wyraźnie odseparowane, to znaczy że żaden ze sobą nie sąsiadował
indeksowo jak to miało miejsce dla architektury sześciokątnej.

    \hypertarget{podsumowanie}{%
\subsection{Podsumowanie}\label{podsumowanie}}

            
        
    

    
    \begin{center}
    \adjustimage{max size={0.9\linewidth}{0.9\paperheight}}{KOHEN_sprawozdanie_Olaf_Werner_291139_files/KOHEN_sprawozdanie_Olaf_Werner_291139_58_2.png}
    \end{center}
    { \hspace*{\fill} \\}
    
    Obie architektury dają bardzo podobne rezultaty po dużej liczbie
iteracji, zaś funkcja sombrero źle się zachowuje. Po za tym dla tego
zbioru danych niektóre klasy są lepiej separowalne od innych. W
przypadku funkcji gaussa nawet w najgorszym przypadku podział był lepszy
niż losowy.

    \hypertarget{uci-har}{%
\subsection{UCI HAR}\label{uci-har}}

    \hypertarget{szeux15bciokux105tna-i-gaussowska}{%
\subsection{Sześciokątna i
gaussowska}\label{szeux15bciokux105tna-i-gaussowska}}

    \hypertarget{klasa-i-bmu-dla-kaux17cdej-obserwacji-testowej}{%
\subsection{Klasa i BMU dla każdej obserwacji
testowej}\label{klasa-i-bmu-dla-kaux17cdej-obserwacji-testowej}}

           
        
    \begin{center}
    \adjustimage{max size={0.9\linewidth}{0.9\paperheight}}{KOHEN_sprawozdanie_Olaf_Werner_291139_files/KOHEN_sprawozdanie_Olaf_Werner_291139_68_1.png}
    \end{center}
    { \hspace*{\fill} \\}
    
    Tutaj mamy bardzo dobrą separację dla leżenia. Jest prawie zupełnie
czysta. Po za tym czynności są mieszane tylko z pewnymi innymi
czynnościami: stanie z siedzeniem, oraz chodzenie w górę w dół i po
prostu.

    \hypertarget{czystoux15bux107-w-zaleux17cnoux15bci-od-neuronu}{%
\subsection{Czystość w zależności od
neuronu}\label{czystoux15bux107-w-zaleux17cnoux15bci-od-neuronu}}

            
        
    \begin{center}
    \adjustimage{max size={0.9\linewidth}{0.9\paperheight}}{KOHEN_sprawozdanie_Olaf_Werner_291139_files/KOHEN_sprawozdanie_Olaf_Werner_291139_71_1.png}
    \end{center}
    { \hspace*{\fill} \\}
    
    Leżenie jest prawie czystę, zaś reszta czynności ma czystość powyżej
0.5. Czystość schodzenia w dół w przypadkach {[}2,2{]} i {[}1,3{]}
wynika z małego rozmiaru klastru.

    \hypertarget{u-macierz}{%
\subsection{U Macierz}\label{u-macierz}}

            
        
    \begin{center}
    \adjustimage{max size={0.9\linewidth}{0.9\paperheight}}{KOHEN_sprawozdanie_Olaf_Werner_291139_files/KOHEN_sprawozdanie_Olaf_Werner_291139_74_1.png}
    \end{center}
    { \hspace*{\fill} \\}
    
    Umacierz pokazuje podobieństwa pomiędzy siedzeniem, a leżeniem i
separację pomiędzy staniem, a schodzeniem w dół. Widzimy też
podobieństwo między schodzeniem w dół, a chodzeniem.

    \hypertarget{szeux15bciokux105tna-i-sombrero}{%
\subsection{Sześciokątna i
sombrero}\label{szeux15bciokux105tna-i-sombrero}}

    \hypertarget{klasa-i-bmu-dla-kaux17cdej-obserwacji-testowej}{%
\subsection{Klasa i BMU dla każdej obserwacji
testowej}\label{klasa-i-bmu-dla-kaux17cdej-obserwacji-testowej}}

            
        
    \begin{center}
    \adjustimage{max size={0.9\linewidth}{0.9\paperheight}}{KOHEN_sprawozdanie_Olaf_Werner_291139_files/KOHEN_sprawozdanie_Olaf_Werner_291139_79_1.png}
    \end{center}
    { \hspace*{\fill} \\}
    
    Zła liczba klastrów. Ale są klastry aktywności ruchowych i pasywnych.
Wykryto też dwa czyste klastry shodzenia w dół. Klaster {[}0,1{]} jest
wymieszany

    \hypertarget{czystoux15bux107-w-zaleux17cnoux15bci-od-neuronu}{%
\subsection{Czystość w zależności od
neuronu}\label{czystoux15bux107-w-zaleux17cnoux15bci-od-neuronu}}

            
        
    \begin{center}
    \adjustimage{max size={0.9\linewidth}{0.9\paperheight}}{KOHEN_sprawozdanie_Olaf_Werner_291139_files/KOHEN_sprawozdanie_Olaf_Werner_291139_82_1.png}
    \end{center}
    { \hspace*{\fill} \\}
    
    Oprócz dwóch małych klastrów schodzenia w dół to pozostałe mają małą
czystość.

    \hypertarget{u-macierz}{%
\subsection{U Macierz}\label{u-macierz}}

            
        
    \begin{center}
    \adjustimage{max size={0.9\linewidth}{0.9\paperheight}}{KOHEN_sprawozdanie_Olaf_Werner_291139_files/KOHEN_sprawozdanie_Olaf_Werner_291139_85_1.png}
    \end{center}
    { \hspace*{\fill} \\}
    
    Klastry są od siebie odległe

    \hypertarget{kwadratowa-i-gaussowska}{%
\subsection{Kwadratowa i gaussowska}\label{kwadratowa-i-gaussowska}}

    \hypertarget{klasa-i-bmu-dla-kaux17cdej-obserwacji-testowej}{%
\subsection{Klasa i BMU dla każdej obserwacji
testowej}\label{klasa-i-bmu-dla-kaux17cdej-obserwacji-testowej}}

            
        
    \begin{center}
    \adjustimage{max size={0.9\linewidth}{0.9\paperheight}}{KOHEN_sprawozdanie_Olaf_Werner_291139_files/KOHEN_sprawozdanie_Olaf_Werner_291139_90_1.png}
    \end{center}
    { \hspace*{\fill} \\}
    
    Podobne wyniki do architektury sześciokątnej. Liczba klastrów jest
jednak nieco inna.

    \hypertarget{czystoux15bux107-w-zaleux17cnoux15bci-od-neuronu}{%
\subsection{Czystość w zależności od
neuronu}\label{czystoux15bux107-w-zaleux17cnoux15bci-od-neuronu}}

            
        
    
   
    
    \begin{center}
    \adjustimage{max size={0.9\linewidth}{0.9\paperheight}}{KOHEN_sprawozdanie_Olaf_Werner_291139_files/KOHEN_sprawozdanie_Olaf_Werner_291139_93_2.png}
    \end{center}
    { \hspace*{\fill} \\}
    
    Leżenie jest prawie czystę, zaś reszta czynności ma czystość powyżej
0.5. Czystość schodzenia w dół w przypadkach {[}2,1{]} i {[}3,1{]}
wynika z małego rozmiaru klastru.

    \hypertarget{u-macierz}{%
\subsection{U Macierz}\label{u-macierz}}

    \begin{center}
    \adjustimage{max size={0.9\linewidth}{0.9\paperheight}}{KOHEN_sprawozdanie_Olaf_Werner_291139_files/KOHEN_sprawozdanie_Olaf_Werner_291139_96_0.png}
    \end{center}
    { \hspace*{\fill} \\}
    
    Siedzenie jest podobne do leżenia, zaś aktywności ruchowe są do siebie
podobne.

    \hypertarget{kwadratowa-i-sombrero}{%
\subsection{Kwadratowa i sombrero}\label{kwadratowa-i-sombrero}}

    \hypertarget{klasa-i-bmu-dla-kaux17cdej-obserwacji-testowej}{%
\subsection{Klasa i BMU dla każdej obserwacji
testowej}\label{klasa-i-bmu-dla-kaux17cdej-obserwacji-testowej}}

            
        
    \begin{center}
    \adjustimage{max size={0.9\linewidth}{0.9\paperheight}}{KOHEN_sprawozdanie_Olaf_Werner_291139_files/KOHEN_sprawozdanie_Olaf_Werner_291139_101_1.png}
    \end{center}
    { \hspace*{\fill} \\}
    
    Mamy 3 klastry: jeden mieszany, jeden aktywności ruchowych i jeden
aktywności pasywnych.

    \hypertarget{czystoux15bux107-w-zaleux17cnoux15bci-od-neuronu}{%
\subsection{Czystość w zależności od
neuronu}\label{czystoux15bux107-w-zaleux17cnoux15bci-od-neuronu}}

           
        
  

    
    \begin{center}
    \adjustimage{max size={0.9\linewidth}{0.9\paperheight}}{KOHEN_sprawozdanie_Olaf_Werner_291139_files/KOHEN_sprawozdanie_Olaf_Werner_291139_104_2.png}
    \end{center}
    { \hspace*{\fill} \\}
    
    Wszystkie mają czystość lepszą od 0.34 ale gorszą od 0.4 co sugeruję że
klasy w klastrach aktywności ruchowych i pasywnych są równomiernie
rozłożone.

    \hypertarget{u-macierz}{%
\subsection{U Macierz}\label{u-macierz}}

    \begin{center}
    \adjustimage{max size={0.9\linewidth}{0.9\paperheight}}{KOHEN_sprawozdanie_Olaf_Werner_291139_files/KOHEN_sprawozdanie_Olaf_Werner_291139_107_0.png}
    \end{center}
    { \hspace*{\fill} \\}
    
    Mamy tu też do czynienia ze zjawiskiem ``odepchnięcia'' tutaj neuronu
{[}2,2{]}. mamy też wyrażny podział między leżeniem, a resztą
aktywności.

    \hypertarget{podsumowanie}{%
\subsection{Podsumowanie}\label{podsumowanie}}

            
        
  

    
    \begin{center}
    \adjustimage{max size={0.9\linewidth}{0.9\paperheight}}{KOHEN_sprawozdanie_Olaf_Werner_291139_files/KOHEN_sprawozdanie_Olaf_Werner_291139_110_2.png}
    \end{center}
    { \hspace*{\fill} \\}
    
    Tutaj w przypadku funkcji gaussa architektura kwadratowa okazała się
lepsza od sześciokątnej. Zaś w przypadku funkcji sombrero jest na
odwrót: architektura sześciokątna okazała się lepsza od kwadratowej. Po
za tym to funkcja gaussa jest zdecydowanie lepsza od sombrero.

    \hypertarget{oguxf3lne-wnioski}{%
\section{Ogólne wnioski}\label{oguxf3lne-wnioski}}

    Funkcja sombrero poprzez możliwość otrzymywania ujemnych wartości może
doprowadzić do dziwnych zachowań, więc odradzam jej używania.
Architektura kwadratowa okazała się nie gorsza od sześciokątnej, ale
ponieważ podania zostały wykonane tylko na dwóch zbiorach danych to
jeszcze nie przesądza o jej wyższości nad sześciokątną. Po za tym
okazało się że są klasy w zbiorach danych które są łatwiej separowalne
od innych i to niezależnie od architektury.


    % Add a bibliography block to the postdoc
    
    
    
    \end{document}
